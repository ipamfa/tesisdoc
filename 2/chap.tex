\chapter{Review of Literature and Studies}
\section{Lattice Theory}
Lattice is algebraic structure just like vector space but with some constraint. In linear algebra we define vector space $ V $ as set of vector with certain dimension such as vector $ \mathbf{x} = (x_1, x_2, ... x_n) $ of dimension $ n $. If we consider vector space $ V $ as set of vector  $\{ ( x_1, x_2, ... x_n ) | x_i \in \mathbb{R} \}$ we can conclude that $ V = \mathbb{R}^n $. For each vector element of vector space, we define two operations: 
\begin{enumerate}
\item additive between vector with another vector
\item multiplicative to scalar
\end{enumerate}
Remind that both operation is associative and distributive when combined. By existence of zero (0) element, identity element (1) and inverse for each operation, we may assume that $V$ is vector space over field $F$ which mean each entry in vector $ x $, $x_i (i=1,2..n)$, is element of $F$ ($x_i \in F$), as we seen before when $ V=\mathbb{R}^n $ then $V$ is vector space over Real field. We also knew that vector space is additive group which generated by set of basis vector $ \{ \mathbf{v}_1, \mathbf{v}_2, ... \mathbf{v}_n \} $ where each $ \mathbf{v}_i \in V $ is linearly independent each other
\subsection{Definition}
A lattice is additive subgroup of vector space $ V $ with discrete characteristic. Suppose $ V $ has dimension  $ n \geq 1 $ and $ V $ is standard real space $\mathbb{R}^n$. If basis set $ B = \{ \mathbf{x}_1, \mathbf{x}_2 ... \mathbf{x}_n \} $ is linearly independent set of vector in $ \mathbb{R}^n $ we define lattice $ L $ generated by basis $B$ (denoted $\mathcal{L}(B)$) as

% lattice definition 1
\[ L = \mathbb{Z}\mathbf{x}_1 + \mathbb{Z}\mathbf{x}_2 +...+ \mathbb{Z}\mathbf{x}_n = \Big \{ \sum_{i=1}^{n} a_{i}\mathbf{x}_{i} | a_{1}, a_{2},...,a_{n} \in \mathbb{Z} \Big \} \]
the main difference between lattice $L$ as vector space with regular standard real space $\mathbb{R}^n$ is lattice $L$ use $\mathbb{Z}$ in scalar multiplicative operation while in $\mathbb{R}^n$ we use $\mathbb{R}$. And because $ \mathbb{Z} \subset \mathbb{R} $ we can say lattice is subspace of real space. We call lattice $L$ is full-rank because the number of basis vector is equal to its space dimension. We may specify lattice in which the number of basis vector is less than space dimension. For example : suppose $ m \leq n $ and 

% lattice definition 2
\[ L' = \mathbb{Z}\mathbf{x}_1 + \mathbb{Z}\mathbf{x}_2 +...+ \mathbb{Z}\mathbf{x}_m = \Big \{ \sum_{i=1}^{m} a_{i}\mathbf{x}_{i} | a_{1}, a_{2},...,a_{m} \in \mathbb{Z} \Big \} \]
Further if we set $B$ as basis in $ \mathbb{Z}^n $ we will get new lattice with each entry is integer so we call $L'$ integer lattice. From now on, if we say lattice this mean to be integer lattice
\subsection{Basis Reduction}
\subsection{Lattice Properties}
\subsubsection{Successive Minima}
In lattice, we denote the shortest vector $\mathbf{v}$ in space spanned by certain basis $B$ where $|\mathbf{v}| \geq |\mathbf{x}_i|$ for each $\mathbf{x}_i \in B$. First minima (basis consist only 1 vector) is the length of vector $\mathbf{x}_{1} \in L$. The second minima (basis consist of 2 vector) is minimum norm value $r$ of vector spanned by basis $\mathbf{x}_{1}, \mathbf{x}_{2} \in L$ such that $|\mathbf{x}_{1}|, |\mathbf{x}_{2}| \leq r_{\min}$. The $i$-th successive minima of $i$ linearly independent vectors $\mathbf{x}_{1}, \mathbf{x}_{2} ... \mathbf{x}_{i}$ is norm value $r$ such that $|\mathbf{x}_{1}|, |\mathbf{x}_{2}| ... |\mathbf{x}_{i}| \leq r_{\min}$. We can express in form of
% successive minima
\[ r = \max \Big ( |\mathbf{x}_{1}|... |\mathbf{x}_{i}|\Big ) \]
if we denote $i$-th successive minima of $L$ by : $\Delta_{i}(L)$ then the following hold:
\[ \Delta_{1}(L) \leq \Delta_{2}(L) ...\leq \Delta_{m}(L) \]
where $m$ is lattice dimension (the number of basis). Hence, the value of successive minima is depend on the choice of basis
\subsubsection{Determinant}
Determinant of lattice $L$ is simply determinant of its basis matrix $B$ : $\det(L)=\det(B)$. If $L$ is not full-rank then determinant is computed by $\det(L)=\det(B.B^{T})$
\subsubsection{Hermite Invariant}
For a given lattice $L$, hermite invariant is defined
% Hermite invariant
\[ \frac{\Delta_{1}(L)^{2}}{(\det(L))^{2/m}} = \Big ( \frac{\Delta_{1}(L)}{(\det(L))^{1/m}} \Big )^{2} \]
where $m$ is lattice dimension (the number of basis)
\subsubsection{Lattice Gap}
Lattice gap is defined by 
\[ \frac{\Delta_{1}(L)}{\Delta_{2}(L)} \]
Lattice gap represents how hard cryptosystem could be break. Both lattice gap and hermite invariant are extensively used in the methods of lattice intersection which was discussed in \cite{Wil12}

\subsection{Lattice Problem}
The most important lattice problems are classical CVP and SVP, from these two problem we can derive other useful lattice problem. But we only discuss the classical problem because those two are most relevant to the context of our thesis
\subsubsection{SVP}
If we revisit the concept of minima, shortest vector problem is equal to : given a lattice $L=\mathcal{L}(B)$ find $\mathbf{v}_1 = \Delta_{1}(L)$, this is the case when basis . For application, it is easier to find approximate SVP which is a SVP with a multiplicative factor $\gamma \geq 1$. Approximate SVP defined as : given lattice $L$ find $u \in L$ such that $ |u| \leq \gamma \Delta_{1}(L) $

\subsection{Sub-Lattice}
Define sub-lattice $M$ is subset of lattice $L$. Suppose $L$ basis is $ \mathbf{x}_{1}, \mathbf{x}_{2} ... \mathbf{x}_{n}$ and $L \subset \mathbb{R}^{n}$. While basis of $M$ is $\mathbf{y}_{1}, \mathbf{y}_{2}, ... \mathbf{y}_{n}$. Basis of $M$ may be spanned by basis of $L$ hence $M \subset L \subset \mathbb{R}^{n}$. We can get sub-lattice by linear transformation from basis to basis
\[ \mathbf{y}_{i} = \sum_{j=1}^{n} c_{ij}\mathbf{x}_{i} \] where $i=1,2...n$ and $c_{ij} \in \mathbb{Z}$. This can be expressed in matrix equation
\[ Y = CX \]
\subsection{Hermite Normal Form}
HNF is form of matrix H $ m x n $ over $\mathbb{Z}$ in which the following conditions are satisfied :
\begin{enumerate}
\item For some $r$ with $ 0 \leq r \leq m $ we have $H_{ij}=0$ for $r \le i \leq m $ and $1 \leq j \leq n $
\item For some $j_{1}, j_{2}..., j_{r}$ with $ 1 \leq j_{1} \le j_{2} ... \le j_{r} \leq n $ we have $H_{ij}=0$ for $ 1 \leq j \le j_{i}$ and $H_{ij_i} \geq 1$
\item For all $i$ and $k$ with $1 \leq k \le i \leq r$ we have $0 \leq H_{kj_i} \le H_{ij_i} $
\end{enumerate}
\subsection{Dual of Lattice}
Given lattice $L$ with basis $B$ dual of L denoted by $L^*$ is computed as $(B.B^{T})^{-1}B$. Dual operation is useful because by using this operation we can define other operation upon lattice including lattice intersection and lattice union
\subsection{Union}

\subsection{Intersection}

\section{Lattice Based Encryption}
\subsection{GGH Cryptosystem}
\subsubsection{Private Basis}
hubungan almost orthogonal basis dan reduced basis $\to$ akhirnya SVP/CVP
\section{Embedding Technique}
\section{Lattice Intersection Attack}

