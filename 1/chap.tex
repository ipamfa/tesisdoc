\chapter{Problem Introduction}
\section{Background}
\par
By the invention of Shor algorithm in 1994 a new threat on number-theoritical cryptosystem has emerged. This algorithm theoritically solve the problem of prime factorization and discrete logarithm on a quantum computer. This is surprising because the decision version of both problem have known to be in NP and co-NP complexity class. Shor algorithm takes only $ O(b^3)$ time and $ O(b)$ space on $b$-bit number input. This algorithm certainly become serious attack for any cryptosystem based on those two problem and lead researchers to find another problem as base of the future cryptosystem

On the other side, since the seminal work of Ajt\"{a}i \textsl{Generating Hard Instances of Lattice Problems} many researcher has attracted on it. The cause of this attention is certain problems on lattice if used as bases of a cryptosystem, would make average-case of this problem equivalent to its worst-case. Simply state, lattice seems as candidate of combinatorics problem suitable to be exploited as next cryptosystem even under availability of quantum computer. There are two group of lattice problem : hard and easy lattice problem. Among many hard lattice problem, the two of classical are most popular : SVP (Shortest Vector Problem) and CVP (Closest Vector Problem) which by van Emde Boas have proven to be NP-hard in 1981 \cite{Ros11}

Unfortunately, strong complexity of its base problem doesnt make the cryptosystem resist to any attack. The first known cryptosystem based on lattice problem is GGH (Goldreich, Goldwasser, Halevi) which proposed in 1997. This system use the term trapdoor function to describe function which work in encryption. The function accept private basis as input. Because of its role in encryption, this function cannot be inverted without key secret. GGH decryption use Babai rounding method as aproximation of CVP problem \cite{Ggh97}. In 1999, Nguyen firstly devised attack on GGH by finding two flaws in GGH : the gap between the successive minima in lattice and the choice of error vector. The first flaw will make CVP instance on lattice built by embedding technique become easier to solve than general CVP. The second flaw will be used in reasoning that at certain degree the probability of one way function is invertible. The next step in Nguyen method is to transform the problem of decryption into CVP problem with smaller error vector. Once the error vector revealed, the message can be retreaved \cite{Ngu99}

After emergence of Nguyen attack, GGH author change its security parameter by increasing lattice dimension from prior 200 to 400. Altough making system more secure, this also make the system more impractical. The next advancement of GGH is made by Micciancio whose main motivation is to improve GGH efficiency without decreasing its security \cite{Mic08}. The method essentially defining new trapdoor function. Unlike trapdoor function of original GGH, new trapdoor function is deterministic and accept basis as its input in Hermite Normal Form (HNF). This advancement increase efficiency of GGH in the form of decreased key size and ciphertext size. Other advancement upon GGH is Pae-Jung-Ha cryptosystem. Almost similar with the previous advancement, main motivation of this cryptosystem is efficiency and practicality. Unlike in Micciancio method where basis is represented in HNF, Pae-Jung-Ha cryptosystem uses the polynomial ring representation instead \cite{Pjh03}. By choosing certain polynomial, we can find various representation of circulant matrix serving as basis and further we can define trapdoor function for encryption and prove it is difficult to invert if secret key absent. Unfortunately, this claim was broken by cryptanalysis work of Han, Kim and Yeom. 

While hard problem of lattice leading to new cryptosystem, the contrary idea of using lattice to attack cryptosystem also work as well. The first who devised this idea is H{\aa}stad who showed that solving simultaneous modular equations in small degree can be done in polynomial time. The main idea of this method is to construct lattice from equations coefficient (which have altered to fit into single chinese remainder theorem equation) and then apply lattice reduction algorithm like LLL to reduce basis of the newly constructed lattice \cite{Has88}. This method can be applied to attack public key cryptogragphy like RSA. If one same message is encrypted and sent to many user, we can take it analagously as solving simultaneous equations (ciphertext sent to many user) which have same solution (same plaintext).

Further enhancement on this method was done by Coppersmith who give different way to build the lattice and develop it to deal with low degree multivariable polynomial equations. But in essential, both method use the similar idea. Another method was developed by Willy and Plantard which use the term lattice intersection \cite{Wil12}. The author of this method claim that lattice intersection simplify the problem of lattice basis reduction. This seamlessly bring us to finding relativelly shortest vector (SVP) in given sublattice. Applying this type of attack on GGH cryptosystem and in the context of choosen ciphertext attack (CCA), the method really invite us to investigate its consequences in the security issue
% Define block styles
\tikzstyle{block} = [rectangle, draw, fill=blue!20, 
    text width=5em, text centered, rounded corners, minimum height=4em]
\tikzstyle{line} = [draw, -latex']
\begin{center}

\begin{tikzpicture}[node distance = 3cm, auto]
	% place nodes
	\node [block, text width=5cm] (step1) {Identifying the weaknesses of lattice based cryptography};
	\node [block, below of=step1, text width=5cm] (step2) {Deepening the basic of lattice-based cryptography};
	\node [block, below of=step2, text width=5cm] (step3) {Deepening the broadcast attack};
	\node [block, below of=step3, text width=5cm] (step4) {Deepening the CCA and CPA attack model};
	
	% draw edges
	\path [line] (step1) -- (step2);
	\path [line] (step2) -- (step3);
	\path [line] (step3) -- (step4);
	
\end{tikzpicture}
 
\end{center}

