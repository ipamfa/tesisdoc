\documentclass[a4paper,11pt,extrafontsizes,oneside,book]{memoir}
% bisa dihemat dg option oldfontcommands
\makeatletter
\DeclareOldFontCommand{\rm}{\normalfont\rmfamily}{\mathrm}
\DeclareOldFontCommand{\sf}{\normalfont\sffamily}{\mathsf}
\DeclareOldFontCommand{\tt}{\normalfont\ttfamily}{\mathtt}
\DeclareOldFontCommand{\bf}{\normalfont\bfseries}{\mathbf}
\DeclareOldFontCommand{\it}{\normalfont\itshape}{\mathit}
\DeclareOldFontCommand{\sl}{\normalfont\slshape}{\@nomath\sl}
\DeclareOldFontCommand{\sc}{\normalfont\scshape}{\@nomath\sc}
\makeatother
\usepackage{amsmath}
\usepackage{natbib}
\usepackage{hyperref}
\usepackage{hypcap}

% Times New Roman package
\usepackage{newtxtext}
\usepackage{newtxmath}
\usepackage{graphicx}
\usepackage{tikz} % diagram framework by Till Tantau
\usetikzlibrary{shapes,arrows}

%\usepackage{afterpage}
\usepackage{rotating} % tabel dlm orientasi landscape
\usepackage{pdfpages} % include pdf file
\usepackage{import}
% arabic
\usepackage{arabtex} 
\usepackage[utf8]{inputenc}
\usepackage[T1]{fontenc}
\usepackage[arabic,english]{babel}

% disini dilakukan definisi2 penyesuaian
\OnehalfSpacing
%\settrims{1cm}{1cm}
%\setheadfoot{}{}

% kita gunakan pdf maka setup
\fixpdflayout

\hypersetup{colorlinks=true,bookmarks=true}
% hook dari memoir ini digunakan untuk mengisi
% logo institusi dan menulis namanya di cover depan
\renewcommand{\maketitlehookd}{%
\vskip 4.5em
\begin{figure}[h] % TODO: insert image disini
\centering
\includegraphics[scale=0.4]{logo-untel}
\end{figure}
\begin{center}
\vfill
\par
\Large{Graduate School of Informatics\\Telkom University}
\selectlanguage{arabic}
\Huge\textsc{اللغة العربية}
\selectlanguage{english}
\end{center}}

% nama referensi 
\renewcommand*{\bibname}{REFERENCES}

% TODO: sttappendix
% lihat page 367 memman
\begin{document}
% setup Margins
%\setlrmargins{}{}{}
\setulmargins{4cm}{*}{*}
\checkandfixthelayout

\frontmatter
%\includepdf{cover}
% preliminary pages sesuai definisi STT Telkom

% Title
\title{Analysis of Broadcast Attack On Lattice Based Cryptography}
\author{Ipam Fuaddina Adam / 2103130003}
\date{December 2015}
\maketitle
\thispagestyle{cleared} % no page number displayed on cover

% Abstract

% Dedication
%\clearpage
%\mbox{}
%\vfill
%\begin{center}
%To my daughter Elmaira and my wife Fauziah Hanum
%\end{center}
%\par
%\vfill
\clearpage

% ToC
\tableofcontents
% Notations
\chapter{Notations}
\begin{tabular}{lrl}
\textbf{\MakeUppercase{SET THEORY}}    & $\cap_{i \in i}S_i$ & intersection of sets\\
\textbf{\MakeUppercase{LATTICE THEORY}}    & $\Delta(L) $ & minima of lattice $L$\\
\end{tabular}


%\setulmargins{1cm}{*}{*}
%\checkandfixthelayout

\mainmatter
\setsecnumdepth{subsection}
\makeheadrule{headings}{\textwidth}{0.8pt}   % berarti page setelah chapter, style-nya headings
\import{1/}{chap}
\import{2/}{chap}

\backmatter
\appendix
\chapter{Example of Broadcast Attack} 

\bibliographystyle{alpha}
\bibliography{ref}
% file ini cuma jadi induk
\end{document}
